\section{The Difference between Intel SGX and ARM TrustZone}\label{sec:difference}
%overview of two technologies 


%introduce what is SGX


%introduce what is TrustZone


%compare them a little bit

\subsection{Overview}
Trusted Computing (TC) is a concept developed by the Trusted Computing Group.
It has been discussed for years. Regarding the meaning of ``trusted'', there
are a lot of controversy. Some people also challenge the idea of Trusted
Computing fundamentally since it may be harmful to keep anonymity over 
Internet and avoid vendor-lock-in. Despite the buzz, some players in 
the technical industry, including Dell, HP, Microsoft, Intel, and ARM 
have made solid progress on realizing this concept.

In this report, we will investigate two concrete Trusted Computing solutions.
They are Intel SGX and ARM TrustZone. 
According to the definitions given by Trusted Computing Group. A complete 
Trusted system must have the following 6 elements: endorsement key, 
secure input and output, memory curtaining, sealed storage, remote 
attestation, and trusted third party. More specifically, both Intel
SGX and ARM TrustZone are two measurements to fulfill some parts
in the whole Trusted system.  
 
In the following subsections, we will first briefly introduce 
Intel SGX and ARM TrustZone respectively. Then we will compare 
them two from certain aspects.

$M=\left [ \begin{array}{lcrr}
	0 & 0 & 0 & 0 \\ 
	0 & 0 & 1 & 0\\
	1 & 0 & 0 & 1\\
	0 & 0 & 0 & 0
\end{array}  \right ]$

\subsection{Intel SGX}
The full name of Intel SGX is Intel Software Guard Extensions. Intel
SGX is a technology to help developers better protect the 
confidentiality and integrity of selected code and data of their 
applications from those rogue software that is running in higher
level including the operating system. It was implemented in 6th
generation Intel Core microprocessors in 2015 for the first time.
Its basic mechanism is to create some protected areas in memory 
which are called ``enclaves''. Developers can explicitly use Intel
SGX SDK during the development to put the data and code of interests 
into enclaves.   
%TODO: watch Stanford video


\subsection{ARM TrustZone}
According to ARM, ARM TrustZone technology ``is a System on
Chip (SoC) and CPU system-wide approach to security''. Its key
idea is to provide two virtual processors to create two 
hardware-separated worlds, the secure world and the insecure
world. Any information flows from the secure world to the 
insecure world must go through the secure monitor. Besides 
this difference, the two worlds have the same capabilities so
each side can operate independently.   

\subsection{Comparison}
\subsubsection{Architecture}
Both Intel SGX and ARM TrustZone are security extensions to 
their existing CPU architectures. Conceptually, Intel SGX
is more lightweight than ARM TrustZone.
Intel SGX essentially is an 
extension to Intel instruction sets. Only with those special 
instructions, data can be read from written to the enclaves.
ARM TrustZone, from our point of view, is a hardware-assisted
virtualization. It provides two completely separate virtual 
running environment. Each side has the complete computation 
capacity. To this extent, Intel SGX and ARM TrustZone have 
different security models.    



\subsubsection{Security}
Intel SGX and ARM TrustZone follows the different security models.
The security provided by Intel SGX are from following two aspects.
First, only several specific instructions, which requires explicitly 
to be called by developers, can access enclaves. Second, the 
cryptographic key is only available to the Intel processors. In short,
Intel SGX tries to help developers to achieve information hiding.
ARM TrustZone, from our perspective, provides isolation, which does
not grant security automatically. The key insights of ARM TrustZone are to
reduce attack surface and isolation. It assumes that developers will 
put almost all code in the insecure world. Those code, such as the 
operating system, is complex, hard to be audited and analysed. They are 
doomed to have exploitable vulnerabilities. On the other hand, even the 
secure world has the capability of running any code, we usually only put 
some small pieces of code in the secure world. Those code has simpler 
logic and performs some critical function, like signing or making a transaction.
So we can review, audit, perform model checking, analyze those code
to make sure it is very likely to be bug free. A malicious user
may temper the operating system in the insecure world easily, however,
he still cannot perform those critical functions in the secure world 
arbitrarily. However, if the code in the secure world has vulnerabilities,
TrustZone cannot prevent a malicious user from exploiting it.             




\subsubsection{Development}
Development in Intel SGX is easier. Intel SGX provides enclave 
binding API in C and C++. To utilize Intel SGX in an application,
a developer can use the API provided by Intel SGX SDK. A developer
can explicitly put the data of interests into enclaves. 
The two most important APIs are for creating and destroying enclaves 
in the memory. Conceptually, they are just like special ``malloc''
and ``free''. ARM TrustZone can be applied to different layers. 
One way is just to use ARM TrustZone APIs to invoke SMC calls in 
application levels. The other way is so called ``co-operative OSes''.
The operating system resides in both the secure world and the 
insecure world. The OS on the two sides must handle interrupts and
scheduling properly. Besides the SMC calls, semaphores, lock-free
algorithms, and shared memory may also be used.       
  

\subsubsection{Invocation}
During runtime, the SGX APIs called by the application will invoke
the driver of SGX to perform encryption, decryption, read, and write
operations. In ARM TrustZone, when the insecure world requests a service
in the secure world, it makes a SMC call to transfer the control to
the secure world.    



\subsubsection{Encryption}
Intel SGX uses symmetric encryption to encrypt the data in 
enclaves. The key will be refreshed in every boot. ARM TrustZone,
on the other hand, does not encrypt the data in the secure world,
since it follows a different security model.      



\subsubsection{Adoption}
WolfSSL is a ``small, portable, embedded, SSL/TLS library targeted
for use by embedded system developers''~\cite{wolfssl}. It supports Intel SGX in
its product. Bromium has prototyped an extension to their product 
which takes advantage of SGX to protect the online credentials~\cite{bromium}.
Samsung Knox~\cite{knox} might be a killer application that utilizes ARM TrustZone
technology. It is widely deployed in Samsung mobile phones. 
%Now Intel SGX SDK supports Microsoft Windows 7, Windows 8.1, Windows 10,
%and Ubuntu 14.04. 
%https://blogs.bromium.com/using-intel-sgx-to-protect-on-line-credentials/


\subsection{Takeaway}
Intel SGX is more lightweight security extension than ARM TrustZone.
They are different regarding their architectures, security models,
ways to develop, and many other aspects. Due to ARM TrustZone is released
earlier than Intel SGX, now ARM TrustZone is being used more widely.   



%Intel SGX is different from ARM TrustZone. Architecturally, with ARM TrustZone, a
%CPU is in two halves which are insecure world and the secure world. Any 
%communication occurs from the insecure world to the secure world is via the 
%Secure Monitor Call (SMC) instruction. At the meanwhile, in Intel SGX model,
%there is only one CPU with many secure enclaves. Conceptually, Intel SGX
%is similar to ``Protected Process'' which is implemented in Microsoft Windows
%Vista for the first time. However, Intel SGX is safer since it is enforced 
%by hardware. 

%Regarding ARM TrustZone, ARM is historically associated with so called ``single-
%purpose systems''. The System on Chip (SoC) is customized to some specific markets
%(e.g. smartphones) so there is only one ``Trust Zone''. Intel SGX has multiple 
%enclaves in a system. So it can be used in a more general multi-purpose chips.     

