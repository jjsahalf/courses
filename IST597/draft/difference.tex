\section{The Difference between Intel SGX and ARM TrustZone}\label{sec:difference}
%overview of two technologies 


%introduce what is SGX


%introduce what is TrustZone


%compare them a little bit

\subsection{Overview}
Trusted Computing (TC) is a concept developed by the Trusted Computing Group.
It has been discussed for years. Regarding the meaning of ``trusted'', there
are a lot of controversy. Some people also challenge the idea of Trusted
Computing fundamentally since it may be harmful to keep anonymity over 
Internet and avoid vendor-lock-in. Despite the buzz, some players in 
the technical industry, including Dell, HP, Microsoft, Intel, and ARM 
have made solid progress on realizing this concept.

In this report, we will investigate two concrete Trusted Computing solutions.
They are Intel SGX and ARM TrustZone. 
According to the definitions given by Trusted Computing Group. A complete 
Trusted system must have the following 6 elements: endorsement key, 
secure input and output, memory curtaining, sealed storage, remote 
attestation, and trusted third party. More specifically, both Intel
SGX and ARM TrustZone are two measurements to fulfill memory curtaining
part in the whole Trusted system.  
 
In the following subsections, we will first briefly introduce 
Intel SGX and ARM TrustZone respectively. Then we will compare 
them two from certain aspects.


\subsection{Intel SGX}
The full name of Intel SGX is Intel Software Guard Extensions. Intel
SGX is a technology to help developers better protect the 
confidentiality and integrity of selected code and data of their 
applications from those rogue software that is running in higher
level including the operating system. It was implemented in 6th
generation Intel Core microprocessors in 2015 for the first time.
Its basic mechanism is to create some protected areas in memory 
which are called ``enclaves''. Developers can explicitly use Intel
SGX SDK during the development to put the data and code of interests 
into enclaves.   
%TODO: watch Stanford video


\subsection{ARM TrustZone}

\subsection{Comparison}
\subsubsection{Architecture}
\subsubsection{Development}
Right now Intel SGX only provides enclave binding API in C and C++.

\subsubsection{Invocation}
\subsubsection{Encryption}
\subsubsection{Security}
\subsubsection{Lifecycle}
\subsubsection{Adoption}
Intel SGX is different from ARM TrustZone. Architecturally, with ARM TrustZone, a
CPU is in two halves which are insecure world and the secure world. Any 
communication occurs from the insecure world to the secure world is via the 
Secure Monitor Call (SMC) instruction. At the meanwhile, in Intel SGX model,
there is only one CPU with many secure enclaves. Conceptually, Intel SGX
is similar to ``Protected Process'' which is implemented in Microsoft Windows
Vista for the first time. However, Intel SGX is safer since it is enforced 
by hardware. 

Regarding ARM TrustZone, ARM is historically associated with so called ``single-
purpose systems''. The System on Chip (SoC) is customized to some specific markets
(e.g. smartphones) so there is only one ``Trust Zone''. Intel SGX has multiple 
enclaves in a system. So it can be used in a more general multi-purpose chips.     

Test~\cite{ChurchEncoding}
