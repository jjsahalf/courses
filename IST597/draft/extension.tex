\section{Extension of ARM TrustZone}\label{sec:extensions}

With respect to the discussed differences between Intel SGX and ARM TrustZone,
 in this section, we presents the extension techniques on TrustZone that allows developers to use it, as in SGX.

To summarize, the main differences of the two techniques are:
\begin{itemize}
\item Architecture: With TrustZone we often think of a CPU which is in two halves i.e. the insecure world and the secure world. In Intel SGX model we have one CPU which can have many secure enclaves (islands).
\item Transparency: TrustZone technology is programmed into the hardware, enabling the protection of memory and peripherals. But SGX is a set of instructions and mechanisms for memory accesses added to Intel architecture processors.
\item Process: Due to the difference in architecture design, the service in Intel SGX have additional verified launching, licensing, and attestations.
\end{itemize}

With respect to these main differences,  we are going to explicitly explain 
the enabling extensions to ARM TrustZone.

\subsection{Isolation}
ARM TrustZone is a hardware isolation technology. Each processor mode has its
own memory access region and privilege. The code running in the normal world
cannot access the memory in the secure world, while the program executed in 
the secure world can access the memory in normal world. The secure and normal 
worlds can be identified by reading the NS bit in the Secure Configuration 
Register (SCR), which can only be modified in the secure world.
Due to the distinction in archietecture, the functionalities are different for 
these two technologies. TrustZone does not distinguish between different 
secure application processes in hardware, in another word, the privilege are
not explicitly specified. 

Each processor mode has its own memory access region and privilege. The code 
running in the normal world cannot access the memory in the secure world, 
while the program executed in the secure world can access the memory in normal
To enable confidentiality between different applications, we should extend the
current system with isolations, either hardware-based or virtually. In 
addition, for specifying priviliges between applications, it requires a
controller which is similar to management engines in Intel SGX. The Intel 
Management Engine (ME) is a micro-computer embedded inside of all recent Intel
processors, and it exists on Intel products including servers, workstations, 
desktops. It will  schedule the application process and control the invocation 
privileges as well. In order to assess security among different application 
processes, it should also provide attestation mechanisms. Local attestation 
enables an process to verify another running on the same physical CPU. Remote 
attestation can be used by a remote party to check if the attested process is 
indeed running on a genuine CPU. It allows initial provisioning of keys and 
secrets. 

\subsection{Abstraction}
The development environment in Intel SGX is more mature than ARM TrustZone,
which provide abundant APIs for developers to use. For developers, Intel SGX 
is more transparent. It is a set of instructions which developers can rely on. 
Developers can take advantage of the enclaves without understanding the 
detailed implementations on hardware enforcement. However, in the design of 
ARM TrustZone, only hardware environment is divided which requires additional
effort of the developer.

To achieve a same development transparency for ARM TrustZone, we should extend 
current design with more abstracted implementation choices. That is, to 
encapsulate possible process initializations, and privilege specifications. By 
abstract the hardware functionalities into software-level, developers can 
easily access a secure development by merely changing the APIs but the service 
logic remains the same and security maintainance is done by the trusted 
computing platform.


\subsection{Additional Processes}
Because of the different archietectures of Intel SGX and ARM TrustZone, the 
processes to set up a secure application and the principle to enforce the 
overall integrity is different. Therefore, to enable a same or similar 
functionality of ARM TrustZone, it requires additional processes in the
development workflow. In this part, we will explain these processes one after
another.

\subsubsection{Verified Process Launch} 
For Intel SGX, when loading an enclave into memory, the CPU measures its 
content in a chained cryptographic hash log, stored in a register called 
MRENCLAVE. This is equivalent to a Platform Configuration Register (PCR) in a 
TPM. Before running the enclave, the CPU verifies MRENCLAVE against a 
vendor-signed version and aborts when finding any mismatches. The verified
enclave launch in Intel SGX enforces the isolation of application processes.
Thus, when isolating different processes in ARM TrustZone, it is still 
required that we extend the verified launching into regular setup procedures.

\subsubsection{Licensing}
SGX has a feature, called launch enclave. During
enclave initialization, the CPU verifies a so-called EINITTOKEN, which 
contains several enclave attributes to enforce, including the debug mode flag. 
The EINITTOKEN has to be signed by a special launch key, which is owned by 
so-called launch enclaves, is sued by Intel. Hence, by issuing proper launch 
enclaves, Intel has full control over which enclaves are to be executed in 
debug or production mode. The SGX evaluation SDK is shipped with a launch 
enclave issuing EINITTOKENs for debug enclaves only. In order to run an 
enclave in production mode, one needs to obtain a proper license from Intel.
This process ensures the confidentiality within any enclaves. Therefore, to
achieve a similar functionality for ARM TrustZone with isolation, we should 
add additional licensing process.

\subsubsection{Attestation}
In Intel SGX, because the running enclave is more than one, the enclave 
contacts the service provider to have its sensitive data provisioned to the 
enclave. The platform produces a secure assertion that identifies the hardware 
environment and the enclave. It has dedicated sealing and attestation 
components that uphold the security model assertions below to prove to the 
service provider that the secrets will be protected according to the intended 
security level. These components enforce a set of designed privileges among
enclaves which is not maintained in ARM TrustZone. Therefore, to ensure the 
security of application processes in ARM TrustZone, we should also maintain
a corresponding attestation mechanism including local attestation and remote
attestation.
